\chapter{Concluding remarks} % Write in your own chapter title
\label{Conclusions}
%\lhead{Chapter~\ref{ChapterLabel}
% \emph{Conclusions}} % Write in your own chapter title to set the page header

Gravitational waves are still awaiting a direct detection and now, with the commissioning of the second generation detectors (Advanced LIGO and Virgo, see Chapter \ref{Chapter One}) just a few years ahead, we have never been closer to this goal and exciting times await us ahead. But, before this happens, now is also the time to further develop the science and analysis techniques we used with the initial LIGO and Virgo towards an emerging gravitational wave astronomy field. The commissioning of Advanced LIGO and Virgo will hopefully give us the first direct detections of GW. The detectors will have a source volume sensitivity three orders of magnitude larger than the initial LIGO and Virgo and a reduced low-frequency cut--off from 40 Hz to 10 Hz. This means that GW from merging binary NS will stay in the detector sensitivity band for 15--20 minutes and GW from merging NS--BH will stay for a few minutes, before coalescing. With this in mind, a new GW--GRB+EM follow--up search could be implemented.

As a final section of this work, we will propose a GW low--latency triggered search that will use gamma-ray burst  candidates, and, in turn, will provide triggers for other EM (radio, X-ray, optical) telescopes for rapid follow-up. This will use the complementarity of GW--GRB searches: since localization with GW is poor, GRB candidates' time and sky location will be used, and, in turn, a GW detection will confirm the compact binary merger and will be readily followed-up in other EM bands, to gain additional source information. This search will make use of the experience we already have from the previous GRB triggered searches and from the recent engineering run and will incorporate a fully-automated GRB candidate database and a fully coherent search method. A GW detection will provide an event of maximum interest to the EM observational community that will be followed--up with minimal delays by a series of telescopes. It will unambiguously confirm the compact binary merger progenitor of a short GRB that, within a day of the burst, might not even be a confirmed GRB yet. Since sky localization using solely the GW detectors is poor (see Chapter \ref{Chapter One} for more details), even with a multi--detector network, we will rely on sky location information from the gamma-ray trigger to alert the EM telescopes. A GW detection triggered by a GRB needs to be followed--up rapidly by EM observations due to the short timescales on which GRB afterglows become extinct (see Chapter \ref{Chapter Two} for more details). Gamma-ray bursts have been used to trigger GW searches for some time (see Chapters \ref{Chapter Four}, \ref{Chapter Six} and \ref{Chapter Seven} for references and example searches) and we will use the expertise of these past searches, only to improve and optimize our effort for Advanced LIGO and Virgo. Thus, we propose a new GRB triggered search strategy: a low-latency fully coherent search that will use triggers fed in real time by GRB detectors and output results on a timescale of hours to days to be followed up by other EM telescopes. Low-latency analysis has been already partially tested during the first LIGO engineering run (ER1, as of June 2012) but it did not include externally triggered searches. There are three main issues to be resolved to be able to implement such a low--latency pipeline: setting up a GRB trigger database that will be updated in real time, adapting the coherent analysis set of codes (used during S6/VSR2 and 3 for GRB-GW searches and for the IPN search, see Chapters \ref{Chapter Four}, \ref{Chapter Six} and \ref{Chapter Seven}) to output results on a much shorter timescale and resolving the submission of GW candidates to EM telescopes for follow-up. Whilst the last issue has been previously tested (e.g. the \emph{Swift} follow--up of two GW candidates \cite{Evans:2012hd}), the first two still need implementation and testing. The first needed element, a full-- and real--time gamma-ray burst trigger database: we propose the use (or the implementation thereof) of a centralized and real-time protocol that will gather gamma-ray burst triggers' information (primary: burst time, duration ($T_{90}$, see Chapter \ref{Chapter Two} for definition) and sky location, secondary: energy and light curve) from different  missions (\emph{Swift}, Fermi--GBM, planned SVOM, etc.). The database needs to be operational full-time to combine different GRB missions' duty cycles and sensitivity restrictions to cover as much live time and sky coverage as possible. Also, the database should record triggers in real--time to allow for a low-latency analysis in the GW data. A good initial candidate for such a tool would be Skyalert (\emph{http://skyalert.org/}) or NASA's GCN circulars, but the specific tools to acquire and rank the triggers have to be developed. A major add--on to this system would be the usage of sub--threshold triggers. Sub--threshold events are usually either stored in the specific detectors' databases (for possible follow-up purposes) without being released publicly or just discarded. They represent the bulk of the electromagnetic output and can provide very useful information to GW data analysts if released. To quote just one example, we used a number of such triggers from \emph{Swift} and Suzaku gamma-ray detectors to refine the localization of some IPN-detected gamma-ray bursts during S5/VSR1. 

A second aspect would be optimizing the templated coherent search method (see Chapters \ref{Chapter Four}, \ref{Chapter Six} and \ref{Chapter Seven} for theory and application and \ref{Chapter Five} for a possible optimization method, the choice of a prior on the GRB inclination angle). This method combines coherently data streams from the individual GW detectors used in the search and provides powerful signal--based veto tests that significantly reduce the number of non--Gaussian noise artifacts. But, given the very low compact binary merger rates, this method needs to allow for a greater detection confidence by a better estimation of the background of false alarms. Detection confidence is given by a better background estimation: the past GW-GRB analyses used the time of the GRB and its sky position to limit the amount of data to be searched over to a short foreground time window, padded on either side by data segments used as background. By computing the background event rate, expected purely from detector noise, we are able to assign a false alarm probability (FAP) to a possible detection event in the foreground window (see Chapter \ref{Chapter Five} for a full analysis). To be a detection candidate, this event must have a very low FAP. The larger the background sample is, the lower the FAP. In order to enlarge the background sample, we implemented a method that time--shifts independent detector data streams and outputs synthetic background segments, for the case of coincident analyses (see Chapter \ref{Chapter Five} for a full methods and analysis description). Given the present compact binary merger event rates within the detectors' range, only by doing time-shifts we could lower the FAP enough to be able to make a detection statement. We plan on implementing this background estimation method in the future GRB follow-up searches for GW; currently the coherent method does not use timeslides. An accurate background estimation method is of maximum importance to be able to make a detection statement and implementing it could give us low enough FAP values for this purpose. Since the timeslides method may have a few pitfalls (e.g. not very efficient when used in conjunction with detectors with very different trigger rates), other background estimation methods will be explored as well.
